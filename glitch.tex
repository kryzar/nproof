\documentclass[a4paper, 12pt]{article}
\usepackage{LoLaTeXpackage}
\usepackage[english]{babel}
\usepackage[utf8]{inputenc}
\usepackage{drinmodmacros}
\usepackage{quiver}
\usepackage{nfproof}

\newcommand{\Fqchi}{\Fq[X, Y]\big/\langle\chi\rangle}
\newcommand{\Fchi}[1]{F_{{#1}}^{(\chi)}}
\DeclareMathOperator{\FracId}{FracId}
\renewcommand{\Pr}{\mathrm{Pr}}
\DeclareMathOperator{\Isom}{Isom}

\title{Yeah it's a group action}

\begin{document}
\maketitle

In all this document, fix $\Fq$ a finite field, $A = \Fq[X]$, $K/\Fq$ a finite
field extension and $\omega \in K$. For every integer $r<1$, let $\Drin{r}(A,
K)$ be the set of rank-$r$ Drinfeld modules $A\to\Ktau$ with constant
coefficient $\omega$. Further fix $\chi \in \Fq[S, T]$ a polynomial that is the
characteristic polynomial of some Drinfeld module $\phi\in\Drin2$ and $R$ the
quotient \[R := \Fqchi,\] equiped with the canonical projection $\pi: \Fq[X,
Y]\to R$.

\begin{definition}
	Denote $\Cchi$ to be set of $\Kbar$-isomorphism classes whose
	representatives are all $K$-isogeneous.
\end{definition}

Our goal is to derive a group action \[\Cl\left(\Fq[X,
Y]\big/\langle\chi\rangle\right) \times \Cchi \to \Cchi.\]

\section{Yeah, it really is a group action}

We already\footnote{See note \texttt{drinmod-torsion-subspace}.} proved the
following result:

\begin{proposition}\label{prop-drinmod-subspace}
	For every Drinfeld module $\phi\in\Drin2$, there exists a Drinfeld module
	$\psi\in\Drin2$and an isogeny $P\in \Kbartau$ from $\phi$ to $\psi$ whose
	kernel is $V$.

    Explicitely, those may be given as \[P = \prod_{v\in V}(X + v) =: \sum_i
    P_i\tau^i\] and $\psi = \sum_{i=0}^r \psi_i \tau^i$ with $\psi_0 = \phi_0$
    and \[\psi_{n+1} = \frac{\sum_{i+j = n+1} P_i \phi_j^{q^i} - \sum_{i+j=n+1
    \atop (i, j) \neq (n+1, 0)} \psi_i P_j^{q^i}}{P_0^{q^{n+1}}}, \quad \forall
    n < r.\]
\end{proposition}

With the notations of \ref{prop-drinmod-subspace}, will denote by $V\cdot\phi$
the Drinfeld module $\psi$, and by $u_{V,\phi}$ the unique unitary isogeny
$\phi\to V\cdot\phi$ given by the proposition.

\paragraph{Isogenies and isomorphisms} Let us begin with a very useful
proposition.

\begin{proposition}\label{prop-isoiso}
	Let two rank-two Drinfeld modules $A\to\Ktau$ that are ordinary are
	$K$-isomorphic \iif they are $\Kbar$-isomorphic and $K$-isogeneous.
\end{proposition}

\begin{nfproof}
	Let $\phi, \psi: A\to\Ktau$ be two Drinfeld modules as in the proposition.
	Of course, if $\phi$ and $\psi$ are $K$-isomorphic, they are
	$\Kbar$-isomorphic and $K$-isogeneous.

	Assume now that $\phi$ and $\psi$ are $\Kbar$-isomorphic and
	$K$-isogeneous. Since those two have rank two and are isomorphic, they have
	height $1$ from \cite[Th. 13.1]{Ros02}. From \cite[Prop.
	4.5.3]{kuhn-mcdoter}, this implies that

	\[\begin{cases}
		\End_K(\phi) = \End_\Kbar(\phi), \\
		\End_K(\psi) = \End_\Kbar(\psi).
	\end{cases}\]

	Now let $\lambda\in \Kbar: \phi\to\psi$ be an isomorphism and $u\in \Ktau:
	\phi\to\psi$ be an isogeny. Then $u\lambda^{-1}$ is a
	$\Kbar$-endomorphism of $\psi$; hence, $u\lambda^{-1}$ lives in $\Ktau$.
	And since $u$ lives in $\Ktau$, $\lambda$ must be in $K$.
\end{nfproof}

\begin{definition}
	For every Drinfeld module $\phi: A\to \Ktau$, denote $\chi(\phi)$ to be its
	characteristic polynomial. If furthermore $\phi$ has rank two, denote
	$\j(\phi)$ to be its $j$-invariant.
\end{definition}

We remind ourselves the structural roles of these quantities.

\begin{proposition}
	Let $\phi,\psi : A\to\Ktau$ be two rank-two Drinfeld modules. Then:
	\begin{itemize}
		\item $\phi$ and $\psi$ are $\Kbar$-isomorphic \iif $\j(\phi) =
			\j(\psi)$.
		\item $\phi$ and $\psi$ are $K$-isogeneous \iif $\chi(\phi) =
			\chi(\psi)$. \qed
	\end{itemize}
	If furthermore $\phi$ and $\psi$ are rank-two and ordinary, then they are
	$K$-isomorphic \iif $(\j(\phi), \chi(\phi)) = (\j(\psi), \chi(\psi))$.
\end{proposition}

\begin{nfproof}
	The first item is \cite[Th. 3.5]{Gek91}, the second is \cite[p.
	286]{caranay-article} and the third is a consequence of the two first and
	of Prop. \ref{prop-isoiso}.
\end{nfproof}

In other words, $K$-isogeny classes are encoded by $\j$-invariants,
$\Kbar$-isomorphism classes are encoded by characteristic polynomials, and
$K$-isomorphism classes are encoded both a $\j$-invariant and a characteristic
polynomial. In this paper, we make no distinction between those classes and
their representations, as well as between classes and representatives.

From Prop. \ref{prop-isoiso}, $\Cchi$ is the set of $\j$-invariants sharing a
same characteristic polynomial.

\paragraph{Construction of the action} To define the group action
\[\Cl\left(A\big/\langle\chi\rangle\right) \times \Cchi \to \Cchi,\] we
iteratively build intermediate functions according to the following "diagram":
% https://q.uiver.app/?q=WzAsNCxbMCwwLCJcXElkXipcXGxlZnQoQVxcYmlnL1xcbGFuZ2xlXFxjaGlcXHJhbmdsZVxccmlnaHQpIFxcdGltZXMgXFxEcmluZGVmIFxcdG8gXFxEcmluZGVmIl0sWzAsMSwiXFxJZF4qXFxsZWZ0KEFcXGJpZy9cXGxhbmdsZVxcY2hpXFxyYW5nbGVcXHJpZ2h0KSBcXHRpbWVzIFxcQ2NoaSBcXHRvIFxcQ2NoaSJdLFswLDIsIlxcRnJhY0lkXFxsZWZ0KEFcXGJpZy9cXGxhbmdsZVxcY2hpXFxyYW5nbGVcXHJpZ2h0KSBcXHRpbWVzIFxcQ2NoaSBcXHRvIFxcQ2NoaSJdLFswLDMsIltcXENsXFxsZWZ0KEFcXGJpZy9cXGxhbmdsZVxcY2hpXFxyYW5nbGVcXHJpZ2h0KSBcXHRpbWVzIFxcQ2NoaSBcXHRvIFxcQ2NoaSJdLFswLDEsIiIsMCx7InN0eWxlIjp7ImJvZHkiOnsibmFtZSI6ImRvdHRlZCJ9fX1dLFsxLDIsIiIsMCx7InN0eWxlIjp7ImJvZHkiOnsibmFtZSI6ImRvdHRlZCJ9fX1dLFsyLDMsIiIsMCx7InN0eWxlIjp7ImJvZHkiOnsibmFtZSI6ImRvdHRlZCJ9fX1dXQ==
\[\begin{tikzcd}
	{\Fchi{1}: \Id^*\left(R\right) \times \Drin2 \to \Drin2} \\
	{\Fchi{2}: \Id^*\left(R\right) \times \Cchi \to \Cchi} \\
	{\Fchi{3}: \FracId\left(R\right) \times \Cchi \to \Cchi} \\
	{\Fchi{4}: \Cl\left(R\right) \times \Cchi \to \Cchi}
	\arrow[dotted, from=1-1, to=2-1]
	\arrow[dotted, from=2-1, to=3-1]
	\arrow[dotted, from=3-1, to=4-1]
\end{tikzcd}\]

Before the actual work, we must define a few notations. In all this paragraph,
fix $I, I'\in\Id^*(R)$ two non-zero ideals and $\phi\in\Drin2$ a Drinfeld
module. 

% \begin{definition}
% 	Let $\pi$ be the canonical projection $\Fq[S, T]\onto\Fqchi$.
% \end{definition}

\begin{definition}
	For every ideal $I\in \Id^*(R)$ and every $\j$-invariant $j\in\Cchi$,
	denote \[\VchiIj := \bigcap_{f\in \pi^{-1}(I)}
	\Ker_\Kbar\left(f\left(\phi_X^{(j)}, \tau_K\right)\right).\]
\end{definition}

Define $\Fchi{1}$ as 

\begin{align*}
	\Fchi{1}: \Id\left(\Fq[S, T]\right) \times \Cchi &\to \Cchi \\
	(I, \phi) &\mapsto \VchiIj \cdot \phi.
\end{align*}

The first step is proving that $\Fchi{1}$ behaves like a group action; this is
Prop. \ref{prop-f1chi-group-action}. To prove this, we need a few lemmas. We
have the following Drinfeld-module-isogeny diagram, with $A, B, C$ being the
unitary isogenies given by Prop. \ref{prop-drinmod-subspace}.

% https://q.uiver.app/?q=WzAsNCxbMCwwLCJqIl0sWzIsMCwiSVxcY2RvdCBqIl0sWzAsMiwiSUknXFxjZG90IGoiXSxbMiwyLCJJJ1xcY2RvdChJXFxjZG90IGopIl0sWzAsMSwiR197SSwgan0gPTogQiJdLFswLDIsIkdfe0lJJywgan0gPTogQSIsMl0sWzEsMywiR197SScsIChJYFxcY2RvdCBqKX0gPTogQyJdXQ==
\[\begin{tikzcd}
	j && {\Fchi{1}\left(I, \phi\right)} \\
	\\
	{\Fchi{1}\left(II', \phi\right)} && {\Fchi{1}\left(I', \Fchi{1}\left(I,
			\phi\right)\right)}
	\arrow["{B}", from=1-1, to=1-3]
	\arrow["{A}"', from=1-1, to=3-1]
	\arrow["{C}", from=1-3, to=3-3]
\end{tikzcd}\]

To prove Prop. \ref{prop-f1chi-group-action}, we prove that the isogenies $CB$
and $A$ are the same --- as they are defined on the same Drinfeld module, they
must have the same image. To do so, we prove that they have the same kernel.

\begin{lemma}\label{lemma-1}
	For every polynomial $f'\in \pi^{-1}(I')$, we have \[f'\left(\Fchi{1}(I,
	\phi)_X, \tau_K\right) B = B f'\left(\phi_X,\tau_K\right).\]
\end{lemma}

\begin{nfproof}
	Write $f' = \sum_{m,n} a_{m,n} X^m Y^n$. Then
	\begin{align*}
		f'\left(\Fchi{1}(I, \phi)_X, \tau_K\right) B
		& = \sum_{m, n} a_{m,n} \left(\Fchi{1}(I, \phi)\right)_X^m {\tau_K}^n
				B \\
		& = \sum_{m, n} a_{m,n} \left(\Fchi{1}(I, \phi)\right)_X^m B {\tau_K}^n
				\eqcomment{, $\tau_K$ commutes with everything} \\
		& = \sum_{m, n} a_{m,n} B \phi_X^m {\tau_K}^n \eqcomment{, $B$ is an
				isogeny} \\
		& = B \sum_{m, n} a_{m,n} \phi_X^m {\tau_K}^n \eqcomment{, $a_{m, n}$'s
				are in $\Fq$ so they commute with everything} \\
		& = B f'\left(\phi_X, \tau_K\right).
	\end{align*}
\end{nfproof}

\begin{lemma}\label{lemma-2}
	For every polynomial $f'\in\pi^{-1}(I')$, we have
	\[\Ker\left(Bf'\left(\phi_X, \tau_K\right)\right) \subset \bigcap_{f \in
	\pi^{-1}(I)} \Ker\left((ff')\left(\phi_X, \tau_K\right)\right).\]
\end{lemma}

\begin{nfproof}
	Straightforward. Let $x\in\Kbar$. We have
	\begin{align*}
		Bf'\left(\phi_X, \tau_K\right)(x) = 0
		& \Rightarrow \forall f\in\pi^{-1}(I), f\left(\phi_X, \tau_K\right) B
				f'\left(\phi_X, \tau_K\right)(x) = 0 \\
		& \Rightarrow \forall f\in\pi^{-1}(I), f\left(\phi_X, \tau_K\right)
				f'\left(\phi_X, \tau_K\right)(x) = 0, \eqcomment{$B$ is a
				generator of $\{f(\phi_X,\tau_K), f\in\pi^{-1}(I)\}$,} \\
		& \Rightarrow \forall f\in\pi^{-1}(I), (ff')\left(\phi_X,
				\tau_K\right)(x) = 0 \eqcomment{, coefficients of $f, f'$ are
				in $\Fq$: they commute with everything}
	\end{align*}
\end{nfproof}

\begin{lemma}\label{lemma-kera-kerbc}
	We have \[\Ker(A) = \Ker(CB).\]
\end{lemma}

\begin{nfproof}
	Double inclusion.

	Let $x\in\Ker(A)$, we want, for every $f'\in \pi^{-1}(I')$, \[x \in
	\Ker\left(f'\left(\Fchi{1}(I, \phi)_X, \tau_K\right)B(x)\right).\] Let then
	$f'\in \pi^{-1}(I')$. From Lemma \ref{lemma-1} we have \[x \in
	\Ker\left(f'(\Fchi{1}(I, \phi)_X, \tau_K)(B(x)\right) \Longleftrightarrow
	\forall f\in \pi^{-1}(I), \Ker\left((ff')\left(\phi_X^{(j)},
	\tau_K\right)\right).\] But $x\in \Ker(A)$, i.e. \[\forall h \in
	\pi^{-1}(II'), \quad x \in \Ker\left(h(\phi_X, \tau_K)\right).\] But for
	every $f\in \pi^{-1}(I)$, $ff' \in \pi^{-1}(II')$\footnote{We have an
	inclusion $\pi^{-1}(I)\pi^{-1}(I') \subset \pi^{-1}(II')$. How do you know
	this is not generally an equality? Well, that's because only one inclusion
	is given in Atiyah-Macdonald.}, which concludes the first inclusion.

	Let $x\in\Ker(CB)$, we want, for every $h\in \pi^{-1}(II')$, \[x \in
	\Ker\left(h\left(\phi_X^{(j)}, \tau_K\right)\right).\] 

	We have \[\pi^{-1}(II') = \pi^{-1}(I)\pi^{-1}(I') +
	\langle\chi\rangle_{\Fq[X, Y]},\] hence \[\bigcap_{h\in\pi^{-1}(II')}
	\Ker\left(h\left(\phi_X, \tau_k\right)\right) =
	\bigcap_{h\in\pi^{-1}(I)\pi^{-1}(I')} \Ker\left(h\left(\phi_X,
	\tau_k\right)\right).\] As $\pi^{-1}(I)\pi^{-1}(I')$ is generated by the
	products $ff', (f, f')\in \pi^{-1}(I)\times \pi^{-1}(I')$, it is sufficient
	to consider every polynomial $h$ written as such a product $ff'$, so that
	we want, for every fixed $f'\in \pi^{-1}(I')$, \[x \in \bigcap_{f\in
	\pi^{-1}(I)} \Ker\left((ff')\left(\phi_X^, \tau_K\right)\right).\] This is
	equivalent to \[x\in\Ker\left(B f'\left(\phi_X^,\tau_K\right)\right)\] from
	Lemma \ref{lemma-1}, which implies \[x\in\Ker\left(f'\left(\Fchi{1}(I,
	\phi)_X, \tau_K\right) B\right)\] from Lemma \ref{lemma-2}. This is true,
	by hypothesis, and we are done.

\end{nfproof}

\begin{proposition}\label{prop-f1chi-group-action}
	We have \[\Fchi{1}\left(II', \phi\right) =
	\Fchi{1}\left(I',\Fchi{1}\left(I, \phi\right)\right).\]
\end{proposition}

\begin{nfproof}
	All the hard work has been made in the lemmas. From Lemma
	\ref{lemma-kera-kerbc}, the Drinfeld module isogenies $CB$ and $A$ defined
	on $j$ are the same. Therefore, the Drinfeld modules they define with Prop.
	\ref{prop-mod-vec-iso} are the same. But those are exactly
	$\Fchi{1}\left(II', j\right)$ and $\Fchi{1}\left(I',\Fchi{1}\left(I,
	j\right)\right)$, finishing the proof.
\end{nfproof}

\begin{proposition}
	We have \[\Fchi{1}(R, \phi) = \phi.\]
\end{proposition}

\begin{nfproof}
	This is because $\mathrm{V}_{R, \phi} = \{0_{\Fq[X, Y]}\}$ and $\{0_{\Fq[X,
	Y]}\}\cdot\phi$ is always\footnote{Ça se vérifie facilement et c'est en
	note de \texttt{drinmod-torsion-subspace}.} $\phi$.
\end{nfproof}

To define $\Fchi2$, we need the following results.

\begin{lemma}\label{lemma-lambda-ker-isomorphism}
	Let $\phi,\phi'\in\Drin2$ be two Drinfeld modules and $\lambda\in\Kbar$ an
	isomorphism $\phi\to\phi'$. Then, for every ideal $I\in\Id^*(R)$, the
	homothetie \[\lambda\Id: V_{I, \phi} \to V_{I, \phi'}\] is a well-defined
	$\Fq$-vector-space isomorphism.
\end{lemma}

\NTS{Définir $\tau_K$}

\begin{nfproof}

	Let $f\in\pi^{-1}(I)$ and write $f = \sum_{m,n} f_{m,n} X^m Y^n$. By
	hypothesis, we have \[\phi_X' = \lambda \phi_X \lambda^{-1},\] so that

	\begin{align*}
		f(\phi_X', \tau_K) &= f(\lambda \phi_X \lambda^{-1}, \tau_K) \\
		&= \sum_{m,n} f_{m,n} (\lambda\phi_X \lambda^{-1})^m \tau_K^n \\
		&= \sum_{m,n} f_{m,n} \lambda\phi_X^m \lambda^{-1} \tau_K^n \\
		&= \lambda \sum_{m,n} f_{m,n} \phi_X^m \lambda^{-1} \tau_K^n \\
		&= \lambda \sum_{m,n} f_{m,n} \phi_X^m \tau_K^n \lambda^{-1} \\
		&= \lambda f(\phi_X, \tau_K) \lambda^{-1} \\
	\end{align*}

	As $\lambda$ is non-zero, the application $\lambda\Id$ is injective, and,
	\[\Ker_\Kbar\left(f(\phi_X', \tau_K)\right) = \Ker_\Kbar(f\left(\phi_X,
	\tau_K)\lambda^{-1}\right).\]

	As $V_{I, \phi} = \bigcap_{f \in \pi^{-1}(I)} \Ker\left(f(\phi_X,
	\tau_K)\right)$, the conclusion now comes easilly.
\end{nfproof}


\begin{proposition}\label{prop-F1chi-iso-Cchi}
	Let $\phi,\phi'\in\Drin2$ be two Drinfeld modules, with characteristic
	polynomial $\chi$, that are $\Kbar$-isomorphic. Then, for every ideal
	$I\in\Id^*(R)$, the images $\Fchi1(I, \phi)$ and $\Fchi1(I, \phi')$ are
	$\Kbar$-isomorphic.

	If $\lambda \in \Kbar$ is the isomorphism $\phi\to\psi'$, then $\lambda^{\#
	V_{I, \phi}}$ is a well-defined isomorphism $\Fchi1(I, \phi)\to\Fchi1(I,
	\phi')$, such that the following diagram commutes:

	\[\begin{tikzcd}
		\phi & {\Fchi1(I,\phi)} \\
		{\phi'} & {\Fchi1(I,\phi')}
		\arrow["u", from=1-1, to=1-2]
		\arrow["\lambda"', from=1-1, to=2-1]
		\arrow["{u'}"', from=2-1, to=2-2]
		\arrow["\lambda^{\# V_{I, \phi}}", from=1-2, to=2-2]
	\end{tikzcd}\]
\end{proposition}

\begin{nfproof}
	Let $I\in\Id^*(R)$. To lighten notations, fix

	\[\begin{cases}
		\psi := \Fchi1(I, \phi) \\
		\psi' := \Fchi1(I, \phi'),
	\end{cases}\]

	let $u, u'$ respectively be the unitary isogenies $\phi\to\psi$,
	$\phi'\to\psi'$ given by Prop. \ref{prop-drinmod-subspace} and fix $V =
	V_{I, \phi}$, $V = V_{I, \phi'}$.

	Some analysis shows that $\lambda^{\# V}$ is the only possible value for
	the desired isomorphism, which we call $\mu$. Indeed, $\mu$ is an isogeny
	$\psi\to\psi'$ \iif\footnote{If $\mu u$ is an isogeny, then $\mu u\phi =
	\mu\psi u$ (as $u$ is an isogeny $\phi\to\psi$). Multiplying on the right
	by the inverse of $u$ in the right-field of fraction of $\Ktau$ (see
	\cite[Section 4.11]{Gos98}) asserts that $\mu$ is an isogeny
	$\psi\to\psi'$. The other implication is trivial.} $\mu u$ is an isogeny
	$\phi\to\psi'$. This is what we prove.

	Recall (see Prop. \ref{prop-drinmod-torsion}) that, as applications, for
	all $x\in\Kbar$ we have \[u(x) = \prod_{v in V}(x + v).\] From Lemma
	\ref{lemma-lambda-ker-isomorphism}, we further have \[u'(x) = \prod_{v \in
	V}(\lambda x + v).\]

	Now, we calculate:
	\begin{align*}
		\mu u \phi_X(x) &= \lambda^{\# V} \prod_{v\in V}(\phi_X(x) + v) \\
		&= \prod_{v\in V} (\lambda \phi_X(x) + \lambda v) \\
		&= \prod_{v'\in V'} (\lambda \phi_X(x) + v') \\
		&= u'(\lambda \phi_X(x)) \\
		&= u'\lambda\phi_X(x) \\
		&= \psi'_Xu'\lambda (x) \\
		&= \psi'_X\left(\prod_{v'\in V'}(\lambda x + v')\right) \\
		&= \psi'_X\left(\prod_{v\in V}(\lambda x + \lambda v)\right) \\
		&= \psi'_X\left(\lambda^{\# V}\prod_{v\in V}(x + v)\right) \\
		&= \psi'_X\left(\lambda^{\# V} u(x)\right) \\
		&= \psi'_X\mu u(x).
	\end{align*}
	This proves that $\mu u$ is an isogeny $\phi\to\psi'$, and we are done.
\end{nfproof}

With this proposition in hand, we can naturally extend $\Fchi1$ as we wish.

\begin{definition}
	Define $\Fchi2$ as
	\begin{align*}
		\Fchi2: \Id^*(R)\times\Cchi &\to\Cchi \\
		(I, j) &\mapsto \j\left(\Fchi1(I, \phi)\right),
	\end{align*}
	where $\phi$ is any Drinfeld-module in $\Drin2$ of $\j$-invariant is $j$.
\end{definition}

This function is well-defined thanks to Prop. \ref{prop-F1chi-iso-Cchi}, and we
have the relation \[\Fchi2(I, \j(\phi)) = \j\left(\Fchi1(I, \phi)\right)\] for
every ideal $I\in\Id^*(R)$ and every Drinfeld module $\phi\in\Drin2$ with
characteristic polynomial $\chi$.

For every $\j$-invariant $j\in K$, there exists a canonical representative in
$\Drin2$:

\begin{definition}
	Given a $j$-invariant $j\in \Kbar$, denote $\phij$ to be the unique
	rank-two Drinfeld module $A\to \Kbartau$ such that 
	\[\phij = 
		\begin{cases}
			\omega + \tau + j^{-1}\tau^2\text{ if } j \neq 0, \\
			\omega + \tau^2 \text{ if } j = 0.
		\end{cases}
	\]
	Now given an ideal $I\in\Id^*(R)$, we note \[V_{I, j} := V_{I, \phij}.\]
\end{definition}

Therefore, given $(I, j)\in \Id^*(R)\times K$, we have \[\Fchi2(I, j) =
\j\left(V_{I, j} \cdot \phij\right).\] The group action-like properties of
$\Fchi1$ are transmitted to $\Fchi2$, in the sense that for every two ideals
$I,I'\in \Id^*(R)$ and every $\j$-invariant $j\in K$, we have

\[\begin{cases}
	\Fchi2(II', j) = \Fchi2(I', \Fchi2(I', j)), \\
	\Fchi2(R, j) = j.
\end{cases}\]

Furthermore, we observe that $\Fchi2$ is trivial for every principal ideal, in
the sense of the following proposition.

\begin{proposition}\label{prop-Fchi2-principal-ideal}
	Let $I\in \Id^*(R)$ and $j\in\Cchi$. If $I$ is principal, then \[\Fchi2(I,
	j) = j.\]
\end{proposition}

\begin{nfproof}
	Let $\pi(\alpha)$ be a generator of $I$. Then we have \[\pi^{-1}(I) =
	\alpha \Fq[X, Y] + \chi \Fq[X, Y\] and therefore 

	\begin{align*}
		V_{I, j} &= \bigcap_{a, b\in \Fq[X, Y]}\Ker_\Kbar\left(a\alpha
				(\phij_X, \tau_K) + b \chi (\phij_X, \tau_K)\right) \\
		&= \bigcap_{a\in\Fq[X, Y]}\Ker_\Kbar\left(a\alpha(\phij_X,
				\tau_K)\right) \\
		&= \Ker_\Kbar\left(\alpha(\phij_X, \tau_K)\right).
	\end{align*}

	If we pick $\alpha$ to be unitary --- which is always possible\footnote{For
	every $\lambda\in\Fq$, $\pi(\lambda \alpha)$ is still a generator of $I$.}
	--- then the isogeny $u: \phij\to \Fchi1(I, \phij)$ given by Prop.
	\ref{prop-drinmod-subspace} is exactly $\alpha(\phij_X, \tau_K)$. But,
	observe that $\alpha$ has coefficients in $\Fq$ and that both $\tau_K$ and
	$\phij_X$ commute with $\phij_X$. This ensures that $\alpha(\phij_X, j)$ is
	an endomorphism of $\phij$. Therefore, \[\Fchi2(I, j) = \j(\Fchi1(I,
	\phi_j)) = \j(\phij) = j.\]
\end{nfproof}

With this property, we are able to do two things, namely: extend $\Fchi2$ to
all the fractional ideals ($\Fchi3$), then defining it ($\Fchi4$) on the class
group of $R$.

\begin{definition}
	Define $\Fchi3$ as
	\begin{align*}
		\Fchi3: \FracId(R)\times\Cchi &\to\Cchi \\
		(r^{-1}I, j) &\mapsto \Fchi2(I, j),
	\end{align*}
	where $r\in R$ and $I\in\Id^*(R)$.
\end{definition}

If we want our action to be trivial on any principal ideal, even fractional,
this is the only possible definition. We must however ensure that this
definition does not depend on the choice of $r$ and $I$, as a fractional ideal
can be represented in multiple ways.

\begin{proposition}
	The function $\Fchi3$ is well defined; it is a group-action, and it is
	trivial on the principal ideals.
\end{proposition}

\begin{nfproof}
	Let $I, I'\in\Id^*(R)$ be two ideals and $r, r'\in\Fq[X, Y]$ be two
	polynomials such that \[r^{-1} I = r'^{-1}I'.\] Fix $j\in\Cchi$. Then $r'I
	= rI'$ and 

	\begin{align*}
		\Fchi2(I, j) &= \Fchi2(r'R, \Fchi2(I, j)) \\
		&= \Fchi2(r'I, j) \\
		&= \Fchi2(rI', j) \\
		&= \Fchi2(rR, \Fchi2(I', j)) \\
		&= \Fchi2(I', j).
	\end{align*}

	The fact that it is a group action has already been discussed, but let's
	review this; fix $j\in\Cchi$. The neutral element of the considered group
	is $R$, and we have \[\Fchi3(R, j) = \Fchi2(R, j) = j.\] Let now $r,
	r'\in\Fq[X, Y]$ and $I, I'\in \Id^*(R)$. Then

	\begin{align*}
		\Fchi3\left(r^{-1}I r'^{-1} I', j\right) &= \Fchi3\left((rr')^{-1}II',
				j\right) \\
		&= \Fchi2(II', j) \\
		&= \Fchi2(I', \Fchi2(I, j)) \\
		&= \Fchi3(r'I', \Fchi3(rI, j)).
	\end{align*}

	The last part is a direct consequence of the definition of $\Fchi3$ w.r.t.
	$\Fchi2$ and of Prop. \ref{prop-F2chi-principal-ideal}. This finishes the
	proof.
\end{nfproof}

With this result, we may finally define $\Fchi4$.

\begin{definition}
	Define $Fchi4$ as 

	\begin{align*}
		\Fchi4: \FracId(R)\times\Cchi &\to\Cchi \\
		(I\mod \Pr(R), j) &\mapsto \Fchi3(I, j).
	\end{align*}
\end{definition}

With this definition and the work we have already done, it is easy to check
that $\Fchi4$ indeed is a group action --- this is left to the reader.

We can now give an explicit formula for our group action. We have:
\begin{align*}
	\Fchi4: \FracId(R)\times\Cchi &\to\Cchi \\
	(r^{-1}I\mod \Pr(R), j) & \mapsto
			\j\left(\left(\bigcap_{f\in\pi^{-1}(I)}\Ker_\Kbar(\phij_X,
			\tau_K)\right)\cdot \phij\right).
\end{align*}

I've seen enough. I'm satisfied.

\section{Omg it's free}

Ok not so fast. We still want our action to be free and transitive. This is
true in some cases.

\begin{lemma}\label{lemma-ordin-end-struct}
	Let $\phi\in\Drin2$ be a rank-two Drinfeld module. If $\chi(\phi)$
	\NTS{verifies something}, then \[\End_\Kbar(\phi) = \Fq[\phi_X, \tau_K].\]
\end{lemma}

\begin{nfproof}
	\NTS{Todo}
\end{nfproof}

\begin{lemma}\label{lemma-ore-ker-div}
	Let $P,Q \in\Ktau$ be two Ore polynomials. Then \[\Ker(P)\subset\Ker(Q)
	\Longleftrightarrow P\mid_\mathrm{R} Q.\]
\end{lemma}

\begin{nfproof}
	\NTS{Todo}
\end{nfproof}

\begin{lemma}
	Let g$\in\Pr(R)$ and $\phi\in\Drin2$ such that $\chi(\phi) = \chi=$. If
	$\chi$ \NTS{verifies something}, then \[g\ast \phi = \phi
	\Longleftrightarrow g = 1.\]
\end{lemma}

\begin{nfproof}

	We will extensively use the following facts: $\deg_\tau(\tau_K) = 2g+1$ and
	$a(\phi_X) = \phi_a$ for any $a\in A$.

	By hypothesis, the isogeny $u_{I, \phi}:\phi\to\phi$ is an endomorphism;
	from Lemma \ref{lemma-ordin-end-struct}, it is therefore of the form
	\[u_{I, \phi} = f(\phi_X, \tau_K)\] for some polynomial $f\in I$. By
	Euclidean division\footnote{With a lexicographic order such that $Y> X$.}
	of this $f$ by $\chi$, we can assume $f$ to be of the form \[f = \alpha(X)
	Y + \beta(X)\] for some polynomials $\alpha, \beta\in\Fq[X]$.

	Under the assumption\footnote{This asumption is dealt with at the end of
	the proof.} that $g$ has a representative $I\in\Id^*(R)$ that is ordinary,
	the polynomial $f$ is, in fact, an invertible constant in $\Fq$. Let $[u,
	v]$ be the Mumford coordinates of $g$. It is crucial to note that we can
	choose those such that

	\[\begin{cases}
		\deg_X(v) \leq \deg_X(u) \leq g, \\
		Y - v(X) \equiv 0 \pmod{u}.
	\end{cases}\]

	From \NTS{some theorem}, the equality \[I = \langle \pi(u(X)), \pi(Y -
	V(X))\rangle,\] holds.

	\begin{nfproof}[($f\in\Fq\times$)]

		We first prove that $\alpha = 0$. As $\Ker(f(\phi_X, \tau_K)) \subset
		\Ker(u(\phi_X))$ because $u\in I$, Lemma \ref{lemma-ore-ker-div}
		ensures that there exists $\xi \in\Ktau$ such that \[u(\phi_X) = \xi
		(\alpha(\phi_X)\tau_K + \beta(\tau_K)).\] Assume that $\alpha\neq 0 $,
		then looking at the $\tau$-degrees of both sides yiels \[\deg_\tau(\xi)
		+ \deg_\tau(\phi_\alpha \tau_K + \phi_\beta) \leq 2 g.\] Note that
		$\phi_\alpha\tau_K$ has a $\tau$-degree $\geq 2g+1$. Therefore, its
		leading $\tau$-term must be 'canceled out' by some $\tau$-term from
		$\phi_v$. This $\tau$-term from $\phi_v$ may only be its leading
		$\tau$-term\footnote{If this is not the case, then $\phi_v$ has an
		$\tau$-term with $\tau$-degree $\geq 2g+1$, which is impossible.}.  As
		this term has even $\tau$-degree, this is impossible. Therefore, our
		assumption that $\alpha\neq 0$ is wrong, and we have \[f(X, Y) =
		\beta(X).\]

		We then prove that $\beta$ is constant and invertible. As previously,
		there exists $\zeta\in\Ktau$ such that \[\tau_K - v(\phi_X) = \zeta
		\beta(\phi_X).\] Given this equation, it is easy\footnote{Just write
		the definitions.} to show that $\zeta$ is an endomorphism of the form
		\[\zeta = \gamma(\phi_X)\tau_K + \delta(\phi_X)\] for some polynomials
		$\gamma,\delta \in \Fq[X]$. Substitating those values and using the
		fact that $\phi: A\to\Ktau$ is a $\Fq$-algebra morphism and that
		$\tau_K$ commutes with any element of $\Ktau$, we have \[\tau_K\phi_{1
		- \gamma\beta} = \phi_{\delta\beta + v}.\] Still by looking at the
		$\tau$-degrees of both sides of the equation, $\phi_{1 - \gamma\beta}$
		must be zero, and therefore $1 - \gamma\beta$ must be zero; $\beta$ has
		no other choice then than to be non-zero. Therefore, we proved that
		\[f(X, Y) \in \Fq.\]
	\end{nfproof}

	This allows us to prove that $I = R$.
	\begin{nfproof}[]
		 To do this, we prove that $u$ is an invertible constant of $\Fq$.
		 Assume that $u$ is a non-constant polynomial. Then it must have an
		 irreducible monic divisor $r\in\Fq[X]$. From \NTS{some theorem}, $r$
		 is coprime to the $A$-characteristic of $K$, and there
		 exists\footnote{V.  notes perso.} a $r$-vector space isomorphism
		 \[\phi[r] \simeq \left(A/rA\right)^2.\] But on this vector space, the
		 equation $\tau_K - \phi_v \neq 0 \pmod{u}$ implies that $v$ is an
		 eigenvalue of $\tau_K$, which is associated to a non-zero eigenvector
		 $x\in\Kbar\times$. This eigenvector verifies

		\[\begin{cases}
			\phi_u(x) = 0, \\
			(\tau_K - \phi_v)(x) = 0.
		\end{cases}\]

		As \[I = \langle \pi(u(X)), \pi(Y - V(X))\rangle\] and \[\Ker(f) =
		\bigcap_{g\in\pi^{-1}(I)}\Ker(g(\phi_X, \tau_K)),\] we necessarily have
		$f(x, \tau_K) = 0$, which is impossible as we proved $f$ to be a
		non-zero constant. Therefore, our assumption that $u$ is non constant
		is false. It is non-zero \NTS{for some reason}. Since it is (trivially)
		contained in $I$, $I$ is therefore $R$.
	\end{nfproof}

	Our work is almost done. To get rid of the hypothesis that $I$ is ordinary,
	just recall that if it is not, we can find $\lambda\in\Fq^\times$ such that
	$\lambda^{-1} I$ is ordinary; but this ideal equals $R$. Therefore, $I =
	\lambda R$, which is obviously principal. And we are done
\end{nfproof}

\begin{proposition}
	If $\chi$ \NTS{verifies something}, then the action $\Fchi4: \Pr(R) \times
	\Cchi \to \Cchi$ is free.
\end{proposition}

\begin{nfproof}
	\NTS{Todo}
\end{nfproof}

\NTS{Finish}

\section{r/traa}

\begin{lemma}\label{l-gcd-phi}
	Let $r\in\N$ and $\phi\in\Drin{r}$ a Drinfeld module. For every polynomials
	$a, b\in A$, we have \[\phi[a]\cap\phi[b] = \phi[\gcd(a, b)].\]
\end{lemma}

\begin{nfproof}
	By double-inclusion. For the first one, by Bézout's theorem, there exists
	polynomials $u,v\in A$ such that $au + bv = \gcd(a, b)$. Then
	\[\phi_{\gcd(a, b)} = \phi_u\phi_a + \phi_v\phi_b,\] from which we
	conclude.

	For the other inclusion, $\gcd(a, b)$ divides both $a$ and $b$, which
	ensures that $\phi_{\gcd(a, b)}$ right-divides both $\phi_a$ and $\phi_b$.
\end{nfproof}

\begin{proposition}
	The group action is transitive.
\end{proposition}

\begin{nfproof}

	We wish to prove that for every two $x, y\in\Cchi$, there exists
	$g\in\Pr(R)$ such that $g\ast x = y$, i.e. that for every two isogeneous
	$\phi,\psi \in\Drin2$, there exists a fractional ideal $I$ of $R$ such that
	such that \[\overline{I} \ast \Isom_K(\phi) = \Isom_K(\psi).\] We prove
	this result with an induction on the $\tau$-degree of the considered
	isogeny.

	If there exists an isogeny $\phi\to\psi$ with $\tau$-degree $0$, then the
	result is true. Such an isogeny is an isomorphism and $0_{\Cl(R)}$ sends
	$\Isom_K(\phi)$ to $\Isom_K(\psi)$.

	Now assume that the result is proved for every integer up-to $d-1$, with
	$d\in\N^*$. Let $u:\phi\to\psi$ be an isogeny with $\tau$-degree $d$; let
	$n\in A$ its norm, i.e. the polynomial in $A$ such that \[\hat{u} u =
	\phi_n.\] The idea of the proof is to obtain "smaller" isogenies
	$u':\phi\to\psi$ that arise from irreducible divisors of $n$. Let $r\in A$
	be an irreducible divisor or $n$; it is easy\footnote{The fact that $V$ is
	an abelian group is obvious, as the addition on $\phi[r]$ is the same as
	that on $\Kbar$. For the scalar-multiplication-stability, let $\overline{f}
	\in \Fq[X]/\langle r\rangle$ and $x\in V$. Then $u (\phi_f(x)) = \psi_f u
	(x) = \psi_f(0) = 0$ and $\psi_r (\psi_f(x)) = \psi_f \psi_r(x) = \psi_f(0)
	= 0$.} to prove that \[V := \Ker(u)\cap\phi[r]\] is a sub-$\Fq[X]/\langle
	r\rangle$-vector space of $\phi[r]$; $\dim(V)$ is either\footnote{The
	$r$-torsion is isomorphic to either $\Fq[X]/\langle r\rangle$ or to
	$\Fq[X]/\langle r\rangle$ depending the $A$-characteristic of $K$.} $0$,
	$1$ or $2$. The result is true in any of those cases.

	\begin{nfproof}[($\dim V = 0$)]
		If $\Ker(\hat{u}) = \phi[n]$, then $u$ has $\tau$-degree $0$, and we
		already solved this case.

		If $\Ker(\hat{u}) \subset\neq \phi[n]$ we can find $x\in\phi[r]$ such
		that $\hat{u}(x) \neq 0$. Then $\hat{u}(x)$ is in $V$ while being
		non-zero. This contradicts the fact that $\dim V$ = 0; contradiction,
		and this sub-case is proved.
	\end{nfproof}

	\begin{nfproof}[($\dim V = 1$)]
		Recall that there exists a Drinfeld module $\psi'\in\Drin2$ and an Ore
		polynomial $u'\in \Ktau$ such that \[u': \phi\to\psi'\] is an isogeny
		and \[\Ker(u') = V.\]

		We wish to complete the following diagram:

		% https://q.uiver.app/?q=WzAsMyxbMCwwLCJcXHBoaSJdLFsyLDAsIlxccHNpIl0sWzEsMSwiXFxwc2knIl0sWzAsMSwidSJdLFswLDIsInUnIiwyXSxbMiwxLCI/IiwxLHsic3R5bGUiOnsiYm9keSI6eyJuYW1lIjoiZG90dGVkIn19fV1d
		\[\begin{tikzcd}
			\phi && \psi \\
			& {\psi'}
			\arrow["u", from=1-1, to=1-3]
			\arrow["{u'}"', from=1-1, to=2-2]
			\arrow["{?}"{description}, dotted, from=2-2, to=1-3]
		\end{tikzcd}\]

		For this, consider the Euclidean division \[u = u'' u' + r,\] with
		$\deg_\tau(r) < \deg_\tau(u')$. However, any element in $V$ cancels $u$
		and $u'$. Therefore, $r(x) = 0$ for any element $x\in V$, yet \[\# V =
		\deg_X(u') > \deg_X(r').\] Therefore, $r$ has no choice but to be
		identically $0$, and the following diagram commutes:

		\[\begin{tikzcd}
			\phi && \psi \\
			& {\psi'}
			\arrow["u", from=1-1, to=1-3]
			\arrow["{u'}"', from=1-1, to=2-2]
			\arrow["{u''}"{description}, from=2-2, to=1-3]
		\end{tikzcd}\]

		From now on, two possibilities: either one of $u'$ or $u''$ has
		$\tau$-degree $0$ and we already solved this case, either both $u'$ and
		$u''$ have a $\tau$-degree $< \deg_\tau(u)$. For this case, we conclude
		by applying the induction hypothesis on both $u'$ and $u''$: they
		respectively give rise to fractional ideals $I'$ and $I''$ such that

		\[\begin{cases}
			\overline{I'} \cdot \Isom_K(\phi) = \Isom_K(\psi'), \\
			\overline{I''} \cdot \Isom_K(\psi') = \Isom_K(\psi).
		\end{cases}\]

		As the map $\cdot$ is a group action, this implies that $I'' I'$ sends
		$\Isom_K(\phi)$ to $\Isom_K(\psi')$, and we are done with this
		sub-case.

	\end{nfproof}

	\begin{nfproof}[($\dim V = 2$)]

		In this case, then \[V = \Ker(u) \cap \phi[r] = \phi[r],\] which
		implies (\ref{lemma-ore-ker-div}) that $\phi_r$ right-divides $u$:
		there exists $u'\in\Ktau$ such that \[u = u'\phi_r.\]

	\end{nfproof}

\end{nfproof}

\end{document}
